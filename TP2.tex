\documentclass{article}
\usepackage{graphicx} % Required for inserting images

\title{TP2 : Tests d’hypothèses}
\author{Oumaima EL BOUROUMI, Matthieu Gomes}
\date{02/02/2025}

\begin{document}

\maketitle



\section{Préparation}

On considère un échantillon \( (X_1, \dots, X_n) \) constitué de \( n \) variables aléatoires indépendantes et identiquement distribuées selon une loi inconnue. Cet échantillon représente les résultats obtenus à un contrôle de synthèse de Statistiques. On en souhaite une étude statistique.

\subsection{1. Coefficient d’asymétrie et kurtosis}

Le \textbf{coefficient d’asymétrie} (ou skewness) mesure la symétrie d’une distribution par rapport à sa moyenne. Il est défini par :

\[
\gamma_1 = \frac{E[(X - \mu)^3]}{\sigma^3}
\]

où :
- \( \mu \) est la moyenne de la distribution,
- \( \sigma \) est l’écart-type,
- \( E[(X - \mu)^3] \) est le moment centré d’ordre 3.

Dans le cas d’une loi normale \( \mathcal{N}(\mu, \sigma^2) \), on peut démontrer que \( \gamma_1 = 0 \), ce qui signifie que la loi normale est parfaitement symétrique.

Le \textbf{coefficient de kurtosis} est donné par :

\[
\gamma_2 = \frac{E[(X - \mu)^4]}{\sigma^4} - 3
\]

où \( 3 \) est soustrait pour que la loi normale ait un kurtosis de 0.

\subsection{2. Test d’adéquation du \( \chi^2 \)}

Nous souhaitons savoir si l’échantillon \( (X_1, \dots, X_n) \) suit une loi normale \( \mathcal{N}(\mu, \sigma^2) \). Nous réalisons un test du \( \chi^2 \).

\begin{enumerate}
    \item[(A)] \textbf{Hypothèses du test}
    \begin{itemize}
        \item \( H_0 \) : L’échantillon suit une loi normale \( \mathcal{N}(\mu, \sigma^2) \).
        \item \( H_1 \) : L’échantillon ne suit pas une loi normale.
    \end{itemize}
    
    \item[(B)] \textbf{Estimation du maximum de vraisemblance}
    
    Les estimateurs du maximum de vraisemblance pour \( \mu \) et \( \sigma^2 \) sont :
    
    \[
    \hat{\mu} = \frac{1}{n} \sum_{i=1}^{n} x_i, \quad
    \hat{\sigma}^2 = \frac{1}{n} \sum_{i=1}^{n} (x_i - \hat{\mu})^2
    \]

    \item[(C)] \textbf{Effectif théorique des classes}
    
    L’effectif théorique dans une classe \( [a, b] \) est donné par :
    
    \[
    N_{[a,b]} = n \times (F(b) - F(a))
    \]

    où \( F \) est la fonction de répartition de la loi normale \( \mathcal{N}(\hat{\mu}, \hat{\sigma}^2) \).

    \item[(D)] \textbf{Condition pour appliquer le test du \( \chi^2 \)}
    
    Pour que le test soit valide :
    \begin{itemize}
        \item Chaque classe doit contenir un effectif théorique d’au moins 5 observations.
        \item Si certaines classes ont un effectif inférieur à 5, il faut les regrouper.
    \end{itemize}
    
    \item[(E)] \textbf{Influence de \( \alpha \) et risque de première espèce}
    
    \( \alpha \) est le seuil de signification du test (\( 5\% \) par exemple), qui correspond au risque de première espèce (rejeter \( H_0 \) alors qu’elle est vraie). Plus \( \alpha \) est grand, plus le test est sensible, mais le risque d’erreur de type I augmente.
\end{enumerate}

\end{document}
